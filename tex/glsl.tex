\documentclass[a4paper,twocolumn]{article}
\usepackage[utf8]{inputenc}
\usepackage{t1enc}
\usepackage[magyar]{babel}
\usepackage{geometry}
\geometry{
 a4paper,
 total={210mm,297mm},
 left=15mm,
 right=15mm,
 top=25mm,
 bottom=15mm,
}              

\usepackage{amsmath}
\usepackage{amssymb}
\frenchspacing

\usepackage{enumitem}
\usepackage{multicol}
\usepackage{calc}
\usepackage{pgf,tikz}
\usetikzlibrary{arrows}
\usetikzlibrary{turtle}
\usetikzlibrary{graphs}
\usetikzlibrary{patterns,snakes}
\usepackage{url}
\usepackage[framemethod=default]{mdframed}

\usepackage{tcolorbox}
\usepackage{minted}
\usepackage{fontawesome}


\newcommand{\degre}{\ensuremath{^\circ}}
\newcommand{\tg}{\mathop{\mathrm{tg}}\nolimits}
\newcommand{\ctg}{\mathop{\mathrm{ctg}}\nolimits}
\newcommand{\arc}{\mathop{\mathrm{arc}}\nolimits}
\renewcommand{\arcsin}{\arc\sin}
\renewcommand{\arccos}{\arc\cos}
\newcommand{\arctg}{\arc\tg}
\newcommand{\arcctg}{\arc\ctg}
\newcommand{\forras}[1]{\hfill\textit{(#1)}}

\newcommand{\progfeladatok}{\textbf{Feladatok} (\faKeyboardO) \smallskip}
\newcommand{\matfeladatok}{\textbf{Feladatok} (\faPencil) \smallskip}
\newenvironment{glsl}[1]
{\VerbatimEnvironment
\begin{tcolorbox}[colback=yellow!5,colframe=yellow!50!black,title={{#1}}]
\begin{minted}{glsl}}
{\end{minted}\end{tcolorbox}}


\newcommand{\glslexample}[2]{
\begin{tcolorbox}[colback=yellow!5,colframe=yellow!50!black,title={#1}]
\inputminted[linenos, xleftmargin=1em]{glsl}{#2}
\end{tcolorbox}
}


\parskip 8pt
\parindent 0pt

\pagestyle{empty}


\begin{document}

\section*{GLSL ES gyorstalpaló}

A GLSL ES az OpenGL specifikáció részét képező árnyaló nyelv egy változata.
Szintakszisa lényegében megegyezik a C programozási nyelv szintakszisával.
Az alábbiakban csak egy kis részhalmazát adjuk meg a nyelnek, ennyi elég a
tábori feladatok megoldásához.

\subsection*{Input értékek}

A ShaderToy több kényelmi szolgáltatást is nyújt az árnyaló nyelv fölé építve. Definiál
például olyan globális értékeket, amelyek a böngészőben futó programban kapnak értéket,
mi pedig használhatjuk őket az árnyalóban. A folytatásban csak ezeket fogjuk használni:

\begin{glsl}{Néhány ShaderToy input változó}
vec3  iResolution; // felbontás pixelben
float iTime;       // eltelt idő mp-ben
vec4  iMouse;      // egér pixel koordináták
\end{glsl}
  
\subsection*{Adattípusok}

A GLSL szigorú a numerikus típusokkal, ha egy \texttt{float} típusú változóba
\texttt{int} értéket próbálunk tölteni, hibát kapunk.

\begin{glsl}{Gyakran használt típusok}
int i, j = 1, 32000;         // egész
float x, y, z = 3.1, 1., .5; // lebegőpontos
vec2  v = vec2(1., 2.);      // vektor
mat2 m = mat2(
    vec2(1., 0.),
    vec2(0., 1.));           // mátrix
                             // oszlopokkal
bool f = x < 10.1;           // logikai 
\end{glsl}

\subsection*{Operátorok}

\subsection*{Gyakran használt függvények}
\texttt{abs, length, dot, cross, normalize, pow, sin, cos, atan}

\texttt{fract, mod, floor, clamp, step, smoothstep}

Adattípusok:
\texttt{int}, 
\texttt{float}, 
\texttt{vec2}, 
\texttt{vec3}, 
\texttt{vec4} 


\subsection*{Vektorok ás mátrixok}

Vektorok koordinátáit többféle módon elérhetjük, sőt egyszerűen permutálhatjuk is.

\begin{glsl}{Swizzling}
vec3 v = vec3(0, 1, 2);
vec2 a = v.xy;           // (0, 1)
vec3 b = v.zzz;          // (2, 2, 2)
vec3 c = v.yzx;          // (1, 2, 0)
\end{glsl}

\subsection*{Vezérlési szerkezetek}
\end{document}