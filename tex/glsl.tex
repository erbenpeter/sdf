\documentclass[a4paper,twocolumn]{article}
\usepackage[utf8]{inputenc}
\usepackage{t1enc}
\usepackage[magyar]{babel}
\usepackage{geometry}
\geometry{
 a4paper,
 total={210mm,297mm},
 left=15mm,
 right=15mm,
 top=25mm,
 bottom=15mm,
}              

\usepackage{amsmath}
\usepackage{amssymb}
\frenchspacing

\usepackage{enumitem}
\usepackage{multicol}
\usepackage{calc}
\usepackage{pgf,tikz}
\usetikzlibrary{arrows}
\usetikzlibrary{turtle}
\usetikzlibrary{graphs}
\usetikzlibrary{patterns,snakes}
\usepackage{url}
\usepackage[framemethod=default]{mdframed}

\usepackage{tcolorbox}
\usepackage{minted}
\usepackage{fontawesome}


\newcommand{\degre}{\ensuremath{^\circ}}
\newcommand{\tg}{\mathop{\mathrm{tg}}\nolimits}
\newcommand{\ctg}{\mathop{\mathrm{ctg}}\nolimits}
\newcommand{\arc}{\mathop{\mathrm{arc}}\nolimits}
\renewcommand{\arcsin}{\arc\sin}
\renewcommand{\arccos}{\arc\cos}
\newcommand{\arctg}{\arc\tg}
\newcommand{\arcctg}{\arc\ctg}
\newcommand{\forras}[1]{\hfill\textit{(#1)}}

\newcommand{\progfeladatok}{\textbf{Feladatok} (\faKeyboardO) \smallskip}
\newcommand{\matfeladatok}{\textbf{Feladatok} (\faPencil) \smallskip}
\newenvironment{glsl}[1]
{\VerbatimEnvironment
\begin{tcolorbox}[colback=yellow!5,colframe=yellow!50!black,title={{#1}}]
\begin{minted}{glsl}}
{\end{minted}\end{tcolorbox}}


\newcommand{\glslexample}[2]{
\begin{tcolorbox}[colback=yellow!5,colframe=yellow!50!black,title={#1}]
\inputminted[linenos, xleftmargin=1em]{glsl}{#2}
\end{tcolorbox}
}


\parskip 8pt
\parindent 0pt

\pagestyle{empty}


\begin{document}

\begin{multicols*}{2}
{\Large \bf GLSL ES gyorstalpaló}

A GLSL ES az OpenGL specifikáció részét képező árnyaló nyelv egy változata.
Szintakszisa lényegében megegyezik a C programozási nyelv szintakszisával.

\textbf{Input értékek}

A ShaderToy definiál néhány globális értéket,
amelyek minden pixelre azonosak, és az árnyalóban használhatjuk őket.

\begin{glsl}{Néhány ShaderToy input változó}
vec3  iResolution; // .xy felbontás pixelben
float iTime;       // eltelt idő mp-ben
vec4  iMouse;      // egér pixel koordináták
\end{glsl}
  
\textbf{Adattípusok}

A GLSL szigorú a numerikus típusokkal, ha egy \texttt{float} típusú változóba
\texttt{int} értéket próbálunk tölteni, hibát kapunk.

\begin{glsl}{Gyakran használt típusok}
int i, j = 1, 32000;         // egész
float x, y, z = 3.1, 1., .5; // lebegőpontos
vec2  v = vec2(1., 2.);      // vektor
mat2 m = mat2(
    vec2(1., 0.),
    vec2(0., 1.));           // mátrix
bool f = x < 10.1;           // logikai 
\end{glsl}

\textbf{Operátorok}

Lebegőpontos számokkal dolgozva egyenlőség
vizsgálat helyett a különbség abszolútértékét
célszerű nézni:\\ $ a = b \Rightarrow \mid a - b \mid < \epsilon$.

\begin{glsl}{Aritmetika}
float y = 3. * 2. + 1. / (3. - 1.); // 6.5
int   e = 11 % 5;                   // 1
float z = x < 0. ? -x : x;
\end{glsl}
    
\begin{glsl}{Logika}
bool b = i < 10 && j < 20
bool c = x != 0. || y != 1.
bool d = !( x >= 100. || y >= 100.)
\end{glsl}

\textbf{Vezérlési szerkezetek}
\begin{glsl}{Ciklus és elágazás}
for(int i = 0; i < 5; i++) {
    v = abs(v) - 1.;
}
if ( w > 0. ) { fg = mix(fg, bg, .5); }
\end{glsl}

\columnbreak
\textbf{Függvények}

\begin{tabular}{ll}
\texttt{abs(x)} & $|x|$\cr
\texttt{sin(x)} & $\sin x$\cr
\texttt{cos(x)} & $\cos x$\cr
\texttt{atan(r)} & $\arctan r$\cr
\texttt{atan(y, x)} & $\in [-\pi, \pi]$, $\underline{v}(x,y)$ szöge \cr
\texttt{pow(x, y)} & $x^y$\cr
\hline
\texttt{length(v)} & $|\underline{v}|$, vektor hossza\cr
\texttt{normalize(v)} & $\underline{v} / |\underline{v}|$, $\underline{v}$ irányú egységvektor\cr
\texttt{dot(v, w)} & $\underline{v}\cdot\underline{w}$, skalárszorzat\cr
\texttt{cross(v, w)} & $\underline{v}\times\underline{w}$, vektoriális szorzat\cr
\hline
\texttt{fract(x)} & törtrész\cr
\texttt{floor(x)} & lefele kerekítés (lebegőpontos)\cr
\texttt{mod(x, y)} & \texttt{x - y * floor(x/y)}\cr
\texttt{clamp(x, a, b)} & \texttt{min(max(x, a), b)}\cr
\texttt{step(edge, x)} & \texttt{x < edge ? 0.0 : 1.0}\cr
\texttt{smoothstep(e1, e2, x)} & sima lépés (lásd jegyzet)\cr
\texttt{mix(x, y, lambda)} & $(1-\lambda)x + \lambda y$\cr
\end{tabular}

\subsubsection*{Vektorok és mátrixok}

Vektorok koordinátáit többféle módon elérhetjük, sőt egyszerűen permutálhatjuk is.

\begin{glsl}{Swizzling}
vec3 v = vec3(0, 1, 2);
vec2 a = v.xy;           // (0, 1)
vec3 b = v.zzz;          // (2, 2, 2)
vec3 c = v.yzx;          // (1, 2, 0)
\end{glsl}

A legtöbb függvény és operátor \emph{vektorizált}, a műveletek
koordinátánként végződnek el.

\begin{glsl}{Vektorizált aritmetika}
vec3 v = vec3(-2.1, 1., -1.);
vec3 p = abs(v); // (2.1, 1., 1.)
vec3 q = v + 1.; // (-1.1, 2., 0.)
\end{glsl}
    

Mátrixok \emph{oszlopovektoraik} felsorolással adhatók meg.
Mátrix és vektor szorzatát a \texttt{*} operátor jelöli.


\begin{glsl}{3D forgatás $Y$ körül}
float a = 3.14 / 10.;
mat3 R = mat3(
    vec3(cos(a), 0., -sin(a)),
    vec3(0.,     1., 0.     ),
    vec3(sin(a), 0., cos(a) ),
); 
vec3 v = vec3(1., 2., 1.);
vec3 w = R * v;    
\end{glsl}
\end{multicols*}

\end{document}