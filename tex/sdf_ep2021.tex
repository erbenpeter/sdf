\documentclass{article}
\usepackage[utf8]{inputenc}
\usepackage{t1enc}
\usepackage[magyar]{babel}
\usepackage{geometry}
\geometry{
 a4paper,
 total={210mm,297mm},
 left=15mm,
 right=15mm,
 top=25mm,
 bottom=15mm,
}            

\usepackage{amsmath}
\usepackage{amssymb}
\frenchspacing

\usepackage{enumitem}
\usepackage{multicol}
\usepackage{calc}
\usepackage{pgf,tikz}
\usetikzlibrary{arrows}
\usetikzlibrary{turtle}
\usetikzlibrary{graphs}
\usetikzlibrary{patterns,snakes}
\usepackage{url}
\usepackage[framemethod=default]{mdframed}

\usepackage{tcolorbox}
\usepackage{minted}


\newcommand{\degre}{\ensuremath{^\circ}}
\newcommand{\tg}{\mathop{\mathrm{tg}}\nolimits}
\newcommand{\ctg}{\mathop{\mathrm{ctg}}\nolimits}
\newcommand{\arc}{\mathop{\mathrm{arc}}\nolimits}
\renewcommand{\arcsin}{\arc\sin}
\renewcommand{\arccos}{\arc\cos}
\newcommand{\arctg}{\arc\tg}
\newcommand{\arcctg}{\arc\ctg}
\newcommand{\forras}[1]{\hfill\textit{(#1)}}

\parskip 8pt
\parindent 0pt

\pagestyle{empty}


\title{Képpont árnyalók és előjeles távolság függvények}
\author{Erben Péter}

\begin{document}
\maketitle

\section{Képpont árnyalók}

\begin{tcolorbox}[title=Képpont árnyalók]
A \emph{képpont árnyalók} (pixel shaders) olyan programok, amelyeket a számítógépi grafikában használnak.
Léteznek szabványos \emph{árnyaló nyelvek} (például GLSL ES), amelyeket a grafikus kártyák
is tudnak értelmezni és futtatni. A futtató hardver vagy szoftver egy téglalap alakú tartomány
minden $(x,y)$ pontjára meghívja a képpont árnyaló programot, aminek egy színt kell visszaadnia,
az $(x,y)$ képpont színét.

A színeket általában RGBA kódolással definiáljuk, vagyis $(r, g, b, a)$ alakú négydimenziós
valós vektorokkal írjuk le, ahol $r$, $g$ és $b$ a vörös, zöld és kék komponens 0 és 1 közé eső
értéke, $a$ pedig az átlátszatlanság.

$$\mathcal{S}: \mathbb{R}^2 \mapsto \mathbb{R}^4$$
$$\mathcal{S}(x, y) = (r, g, b, a)$$
\end{tcolorbox}

A képpont árnyaló programok egyik hasznos jellemzője, hogy a készülő kép minden képpontját
egymástól függetlenül színezik ki. Ez jelentősen tudja javítani például számítógépes animációk
sebességét (a másodpercenként kirajzolható képkockák számát), ha a grafikus hardver támogatja
a képpont árnyaló program \emph{párhuzamosított} futtatását.

\subsection{ShaderToy}

Az egyik népszerű online alkalmazás képpont árnyékolók tanulásához az Inigo Quilez által fejlesztett
\url{shadertoy.com}. A következőkben a ShaderToy által használt képpont árnyaló nyelvet fogjuk használni,
ami a GLSL ES egy változata.

\subsubsection{Hello, ShaderToy!}

Az árnyaló nyelv hasonlít a C programozási nyelvhez (ha ez segít valakinek). Az $\mathcal{S}(x,y)$
függvényt a \texttt{mainImage} függvény valósítja meg. A bemenő képpont (pixel) $x$ és $y$ koordinátája
a bemenő \texttt{fragCoord} objektumból olvasható ki: \texttt{fragCoord.x} és \texttt{fragCoord.y}.
A kimenő szín-vektort a \texttt{fragColor} változóban kell kiszámolni.

\begin{tcolorbox}[colback=yellow!5,colframe=yellow!50!black,title=Példa: ferde négyzetet rajzoló árnyaló]
\begin{minted}[linenos, xleftmargin=1em]{glsl}
void mainImage( out vec4 fragColor, in vec2 fragCoord )
{
  vec2 uv = (fragCoord - 0.5 * iResolution.xy) / iResolution.y;
  vec3 rgb = abs(uv.x) + abs(uv.y) < 0.2 ? vec3(0.0) : vec3(1.0);
  fragColor = vec4(vec3(uv.x), 1.0);
}
\end{minted}  
\end{tcolorbox}

A fenti példa egy szokásos transzformációt is elvégez, mielőtt a tényleges árnyaló logikát megvalósítaná.

A 3. sorban a globálisan elérhető \texttt{iResolution} objektum (vektor) felhasználásával eltolja a $(0, 0)$ pontot
a kép középpontjába, majd normalizálja a koordinátákat úgy, hogy az $y$ érték $-0.5$ és $0.5$ közé essen.
(Itt láthatjuk, hogy a legtöbb művelet az árnyaló nyelvekben ,,vektorizált'', a műveletek értik, hogy amikor
vektorokkal dolgoznak, koordinátánként kell-e elvégezni a számolást.)

A 4. sorban feketére vagy fehérre színezzük az $(x, y)$ képpontot attól függöen, hogy teljesül-e
az $|x|+|y| < 0.2$ egyenlőtlenség. (A fekete szín RGB kódolása három nulla, a \texttt{vec3} \emph{konstruktor}
érti, hogy ha egy paramétert kapott, akkor mindhárom koordinátát arra állítja be.)

Végül az 5. sorban az látható, hogy a ShaderToy aktuális verziójában az $a$ értéke mindig 1.0, vagyis
nem változtatható az átlátszatlanság.


\subsubsection{Input értékek}

\begin{tcolorbox}[colback=yellow!5,colframe=yellow!50!black,title=Néhány ShaderToy input változó]
\begin{minted}{glsl}
vec3      iResolution;      // viewport resolution (in pixels)
float     iTime;            // shader playback time (in seconds)
int       iFrame;           // shader playback frame
vec4      iMouse;           // mouse pixel coords. xy: current (if MLB down), zw: click
\end{minted}  
\end{tcolorbox}
  

\subsubsection{Típusok}

\subsubsection{Operátorok}

\subsection{Bevezető feladatok}

\begin{enumerate}
  \item Rajzoljunk szürkeátmenetet vízszintesen, függőlegesen és körkörösen.
  \item Rajzoljunk adott méretű négyzetet origó középponttal.
  \item Rajzoljunk adott sugarú kört origó középponttal.
  \item Készítsünk egyszerű függvényábrázoló árnyékolót $y = f(x)$ alakban adott függvények kirajzolásához.
\end{enumerate}

\subsection{Animáció}

Az árnyékolók nagyon egyszerűen használhatók animációk programozására, csupán
be kell vezetnünk egy idő input változót:

$$\mathcal{S}(x, y, t) = (r, g, b, a)$$

A Shadertoy esetében ez az input változó az \texttt{iTime}, ami másodpercben méri az animáció indítása
óta eltelt időt. Egy képkocka kiszámítása közben minden árnyaló hívás azonos $t$ értéket kap.

Az alábbi példában a kirajzolt kör sugara az eltelt időtől függően változik, és a kör középpontja
egér-kattintással módosítható.

\begin{tcolorbox}[colback=yellow!5,colframe=yellow!50!black,title=Változó sugarú és középpontú kör]
\begin{minted}{glsl}
float circle(vec2 p, vec2 o, float r)
{
  return length(p - o) - r;
}
    
vec4 color(in float d)
{
  float baseColor = d < 0.0 ? 0.5 : 1.0;
  return vec4(vec3(baseColor), 1.0);
}
    
vec2 toUV(vec2 coord)
{
  return (coord - 0.5 * iResolution.xy) / iResolution.y;
}
    
void mainImage( out vec4 fragColor, in vec2 fragCoord )
{
  vec2 uv = toUV(fragCoord);
  vec2 mouse = toUV(iMouse.xy);
  float d = circle(uv, mouse, fract(iTime) * 0.3);
  fragColor = color(d);    
}
\end{minted}  
\end{tcolorbox}

\subsection{Geometriai transzformációk}

Egy $g$ síkbeli geometriai transzformáció a sík minden pontjához a sík valamelyik pontját rendeli.

$$g(x, y) = (x', y')$$

\textbf{Feladat}: Adott egy síkbeli $\mathcal{A}$ alakzat az $\mathcal{S}()$ árnyalóval leírva. Hogyan állítható elő
az $\mathcal{A}' = g(\mathcal{A})$ alakzat árnyalója?

\textbf{Feladat}: Adjuk meg (és próbáljuk ki) a következő transzformációkat:
(a) eltolás adott vektorral;
(b) origó körüli forgatás;
(c) origó középpontú nagyítás;
(d) origó középpontú, egységsugarú inverzió.


\subsection{Alakzat ismétlése}

Ha van egy árnyalóval definiált alakzatunk, egyszerűen sokszorozhatjuk meg az alakzatot.

\subsection{Összetettebb feladatok}

\begin{enumerate}
  \item Rajzoljunk sakktáblát, majd animáljuk (eltolás, elforgatás).
  \item Rajzoljunk négyzetrácsot. (Itt csak a rácsegyenesek látszanak, a négyzetek belseje nincs kiszínezve.)
  \item Rajzoljunk hatszögrácsot.
\end{enumerate}

\section{Előjeles távolság függvények a síkon}

\begin{tcolorbox}[title=Előjeles távolság függvény (signed distance function)]
  Az \emph{előjeles távolság függvény} egy $(x,y)$ pont előjeles távolságát adha meg egy ponthalmaz határától mérve.
  Az egyszerűség kedvéért feltehetjük, hogy az alakzat egy zárt, folytonos, nem önmetsző görbével van megadva.
  $$ t: \mathbb{R}^2 \mapsto \mathbb{R}$$
  $$ t(x, y) = d$$

  Ha $(x,y)$ rajta van az alakzat határán, akkor $t = 0$. Külső pontokra $t$ a pont és az alakzat távolságát adja meg.
  Belső pontokra $t$ negatív előjellel adja meg a pont és a határgörbe távolságát.
  
  Az előjeles távolság függvények meglepően jól használhatók arra, hogy egy képet vagy animációt felépítsünk egyszerű
  alap-alakzatokból.
  \end{tcolorbox}

\subsection{Bevezető feladatok: egyszerű alakzatok előjeles távolság függvénye}

Adjuk meg a következő alakzatok előjeles távolság függvényét:

\begin{enumerate}
  \item Origó középpontú, $r$ sugarú kör.
  \item Origó középpontú, adoot méretű, tengely-párhuzamos téglalap.
  \item Végpontjaival adott szakasz.
  \item (*) Szabályos sokszög.
\end{enumerate}

\subsection{Színezési modellek}

\subsection{Polár-koordináták}

\subsection{Kompozíció}

\subsection{Deformáció}

\subsection{Összetettebb feladatok}

\includegraphics[width=5cm]{images/mug.png}
\includegraphics[width=5cm]{images/cogwheels.png}

\section{Előjeles távolság függvények a térben}

\subsection{Sugár séta algoritmus (ray marching)}

\subsection{Egyszerű megvilágítási modell}

\subsection{Feladatok}


\end{document}
